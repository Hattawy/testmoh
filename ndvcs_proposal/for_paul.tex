Measuring Deeply Virtual Compton Scattering on the neutron (n-DVCS) is one of the necessary steps to complete our understanding of the structure of the nucleon in terms of Generalized Parton Distributions . n-DVCS allows to operate a flavor decomposition of the GPDs and plays a complementary role to DVCS on a transversely polarized proton target in the determination of the GPD $E$, the least known and least constrained GPD that enters Ji's angular momentum sum rule. To start the experimental program of DVCS on the neutron, we propose to measure beam-spin asymmetries (BSA) for n-DVCS ($ed\to en\gamma(p)$) with the upgraded 11-GeV CEBAF polarized-electron beam and the CLAS12 detector. The sensitivity of this observable to the GPD $E$ is expected to be maximal for values of $(Q^2, x_B)$ which are attainable only with an 11-GeV beam. 
For the detection of the recoil neutron, necessary to ensure the exclusivity of the reaction after having detected the scattered electron and the DVCS photon, we are constructing a scintillator-barrel detector to be placed in the Central Detector, between the CTOF and the solenoid magnet (the n-DVCS neutrons are in fact expected to be emitted mostly at backward angles). The Central Neutron Detector (CND) will be made of three radial layers of scintillator paddles (48 paddles per layer), coupled two-by-two upstream with semi-circular light guides and read downstream by photomultipliers placed outside of the high magnetic-field region and connected to the bars via 1.5-m-long bent light guides. Our GEANT4-based simulations, calibrated with measurements in cosmic rays carried out on a prototype, show that the efficiencies and resolutions obtainable with this detector, as well as its photon-rejection capabilities, match the requirements of the experiment. 
In order to provide an accurate mapping of the n-DVCS beam-spin asymmetry over the available 4-dimensional ($Q^2$, $x_B$, $-t$, $\phi$) phase space, we request 90 days of running on a liquid deuterium target with the maximum available beam energy, 11 GeV, and 85\% of beam polarization. 
