\section{Beam-time request}

We request 50 new days of beam time for production running on the $^{14}$ND$_3$ target with an 11-GeV polarized electron beam, at 10 nA of current. These days will be added to the 50 already allocated for Run Group Cb. 
%This is necessary to obtain small enough error bars, for the two polarized-target observables, to allow us to bin them as finely as the BSA that will be extracted from the unpolarized-nDVCS experiment \cite{proposal}, and thus manage the CFF extraction by simultaneous fitting of the three observables.  
In order to acquire the roughly 10\% of counts on $^{12}$C that are necessary to estimate the dilution factor (Section~\ref{sec_dilution}) and to remove the nuclear background when studying the exclusivity cuts (Section \ref{sec_excl_cuts}), and given the maximum tolerable luminosity of CLAS12 of $10^{35}$ cm$^{-2}$s$^{-1}$, we will need a total of 10 days of running on a 2-cm-long $^{12}$C target, also with a beam intensity of 10 nA. 
Twelve days will be spent, with and without beam, in target-related operation, as explained in Section~\ref{sec_target_over}. 
Including 3 days of Moeller runs to monitor the beam polarization the whole experiment, the part already approved plus the extension, will take 125 days for completion. 

%We anticipate that approximately half of this request can be accumulated in parallel with other approved experiments utilizing the polarized ND$_3$ target (see Section~\ref{the_snake}).  We therefore request 50 new days of production running on ND3, combined with 12 days for ancillary measurements and overhead. 

\begin{table}
\begin{center}
\begin{tabular}{|c||c|}
%\hline
%Testing and commissioning & 2 days\\
\hline
Production data taking at $10^{35}$ cm$^{-2}$s$^{-1}$ on ND$_3$ & 100 days (50 are already approved)\\
\hline
Target work & 12 days\\
\hline
Production data taking on $^{12}$C target & 10 days\\
\hline 
Moeller polarimeter runs & 3 days\\
\hline
\hline
Total beam time request & 125 days\\
\hline
\end{tabular}
\caption{Beam-time request for the extension of Run-group Cb. }
\label{beam_time}
\end{center}
\end{table}

%\section{Compatibility with other CLAS12 experiments/run groups}\label{the_snake}
%The collaboration recognizes that beam time during the 12 GeV era is a highly
%precious commodity, and that our request for 100+ days of this commodity must be
%well justified. As such, we reiterate that these first time measurements of
%target- and double-spin asymmetries in nDVCS, in tandem with the BSA measurements
%of E12-11-003, are absolutely necessary to make the first ever model-independent
%extractions of neutron Compton Form Factors, which will in turn allow first-ever flavor-separation of CFFs. The aforementioned experiment was granted 90 days from PAC38 and was later recognized as "High Impact" by PAC41. We respectfully suggest the present proposal has similar merit, and deserves the 100 days we request. Only in this way can the precision of its TSA and DSA results fully
%compliment the BSA measurements of E12-11-003, and permit a meaningful extraction of the neutron CFFs. 

%To make optimal use of the CEBAF12 beam, however, we anticipate that
%approximately half of our request can be accumulated in parallel with other
%approved experiments utilizing the polarized ND$_3$ target (E12-006-109,
%E12-007-107, and E12-09-007b) \cite{schedule}. We are therefore requesting 
%50 new days of production running on ND$_3$, combined with 12 days for ancillary 
%measurements and overhead.

\section{Conclusions}
Our knowledge of the three-dimensional structure of the nucleon has become richer in the last few years thanks to the introduction of the formalism of the Generalized Parton Distributions and to the subsequent wealth of experimental results on Deeply Virtual Compton Scattering which have recently become available. After the pioneering experimental results on DVCS, which raised the interest in this reaction as a means to achieve a tomographic description of the nucleon, it became evident, thanks to the analysis of the second generation of proton-DVCS dedicated experiments and to the advancement in the theory and phenomenology of GPDs, how only the combined measurement of several DVCS observables in a vast kinematic space can allow one to disentangle the contributions of the various GPDs and their complex kinematic dependences. While our knowledge of the three-dimensional structure of the proton is progressing considerably - the first attempts at its tomographic description have recently been made thanks to CLAS data taken at 6 GeV \cite{pisano,hs}, and a vast experimental program of pDVCS is planned for JLab at 12 GeV - neutron GPDs remain a mostly virgin field at this stage. The importance of extracing neutron CFFs is paramount if we want to ultimately perform a flavor decomposition of the GPDs. 
We propose here to make the first ever nDVCS measurements of spin observables, target- and double-spin asymmetries, with a polarized target.  We view the experiment as complementary to E12-11-003, which will measure the beam-spin asymmetries for nDVCS at the same kinematic points, and which is currently listed as a "high-impact" 12 GeV experiment. 
If suitable shieldings are found to allow its use with a rastered beam, the detector system will include the Forward Tagger, added to the standard CLAS12 configuration. The polarized target is already being developed and will be used also for other CLAS12 experiments. The expected statistical precision and coverage for TSA and DSA that can be achieved with 100 days of beam time will allow us to extract, fitting them together with the BSA from E12-11-00, various neutron Compton Form Factors in a model-independent way. Quark-flavor separation will be obtained by the linear combination of these neutron CFFs with the proton CFFs extracted from the pDVCS CLAS12 experiments. 
