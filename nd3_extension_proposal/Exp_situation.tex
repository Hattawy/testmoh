\section{Experimental situation}\label{sec_exp_situation}
The determination of all the GPDs is clearly a non-trivial task, and requires measurement of several observables on both proton and neutron targets. 
Such a dedicated experimental program, concentrating on a proton target, has started worldwide in the past few years. 
Table~\ref{dvcs_exp_summary} summarizes the current situation.  It is evident that while data exist for all proton observables, neutron DVCS data is woefully lacking.  The only existing nDVCS experiment was performed in Hall A \cite{malek}, where the beam-polarized cross section difference was extracted, albeit with small kinematical coverage, low statistical precision, and high systematic uncertainties. There also exists a number of approved 12 GeV pDVCS experiments at JLab, both in Hall A and Hall B, but only one approved neutron experiment, to measure the beam-spin asymmetries using CLAS12 \cite{proposal}.  While the new pDVCS experiments will greatly increase both the coverage and statistics of the existing proton data, we propose to further advance the nDVCS program by performing the first ever measurements of target-spin and double-spin asymmetries on a longitudinally polarized neutron target.

\begin{savenotes}
\begin{table}[t]
   \centering
   \begin{tabular}{|c||c|c|c|} 
\hline
Observable & Sensitivity & Completed & 12-GeV \\
(target) & to CFFs & experiments & experiments \\
\hline
\hline
    $\Delta\sigma_{beam}$(p) & $\Im{\rm m}{\cal H}_p$ & Hall A \cite{carlos},\cite{E07007}\footnote{Analysis underway.}, CLAS \cite{hs} &   Hall A \cite{E1206114}, CLAS12 \cite{E1206119}\\ 
& & &Hall C \cite{E1213010}\\

\hline
    BSA(p)     &    $\Im{\rm m}{\cal H}_p$       & HERMES \cite{hermes}, CLAS \cite{stepan,fx,pisano}   & CLAS12 \cite{E1206119} \\
\hline 
   TSA(p)   &    $\Im{\rm m}\widetilde{\cal H}_p,\Im{\rm m}{\cal H}_p,$     &  HERMES \cite{hermes}, CLAS \cite{shifeng,erin,pisano} & CLAS12 \cite{E1206119}\\
\hline 
    DSA(p)    &    $\Re{\rm e}\widetilde{\cal H}_p,\Re{\rm e}{\cal H}_p$     &  HERMES \cite{hermes}, CLAS \cite{pisano}& CLAS12 \cite{E1206119}\\
\hline 
    tTSA(p)    &    $\Im{\rm m}{\cal H}_p,\Im{\rm m}{\cal E}_p$               &  HERMES \cite{hermes} &  CLAS12 \cite{E1212010}\\
\hline
    $\Delta\sigma_{beam}$(n)    &    $\Im{\rm m}{\cal E}_n$       &  Hall A \cite{malek},\cite{E08025}\footnotemark[\value{footnote}] & \\
\hline
    BSA(n)    &    $\Im{\rm m}{\cal E}_n$       &   & CLAS12 \cite{proposal}\\
\hline
   \end{tabular}
   \caption[Summary of existing and proposed DVCS experiments]
   {Summary of all existing data on proton and neutron DVCS spin observables, along with their sensitivity to the various GPDs. The ``t'' prefix indicates transversely polarized target.}\label{dvcs_exp_summary}
\end{table}
\end{savenotes}

The currently approved CLAS12 program includes about 120 days of beam time allocated for data taking on unpolarized proton target, 120 days on longitudinally polarized proton target (NH$_3$), 90 days on unpolarized deuterium target, and 50 days (plus 15 of overhead) on longitudinally polarized deuterium target (ND$_3$). Considering that the polarization of the deuteron (and of the neutron) in ND$_3$ is about half that of the proton in NH$_3$, that the cross section for neutron-DVCS is more than a factor of two smaller than the one for proton-DVCS, and that the detection efficiency for neutrons is at least a third of that for protons, as of today there is a big difference in statistical power between the polarized proton and neutron datasets for CLAS12. Doubling the current statistics on (ND$_3$), as this proposal aims to do, would contribute to reducing this gap. 
